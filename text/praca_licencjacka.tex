% Sprawdzić tekst pod kątem napisu TODO
\documentclass[declaration,shortabstract,masc]{iithesis}

\usepackage[utf8]{inputenc}
\usepackage{datetime}
\newcommand{\cpp}{C\raisebox{0.5ex}{{\tiny\textbf{++}}}}

\polishtitle{Analiza teoretyczna i porównawcza\fmlinebreak efektywnej pamięciowo, wysoce wydajnej\fmlinebreak tablicy asocjacyjnej SILT\fmlinebreak (Small Index Large Table)}
\englishtitle{Theoretical and Comparative Analysis of\fmlinebreak Memory-Efficient, High-Performance\fmlinebreak Associative Array SILT\fmlinebreak (Small Index Large Table)}
\polishabstract{SILT (Small Index Large Table) jest efektywną pamięciowo, wysoce wydajną tablicą asocjacyjną. Stanowi alternatywę dla innych popularnych struktur danych opracowanych na potrzeby rozwiązania problemu przechowywania par klucz-wartość. Została utworzona z myślą o ograniczeniu i zbalansowaniu zużycia pamięci oraz czasu obliczeń. W swojej pracy chcę się skupić na analizie teoretycznej tej struktury danych oraz na porównaniu jej działania z dostępną w bibliotece standardowej języka \cpp\ funkcją \texttt{std::unordered\_map}. W ramach analizy opiszę konstrukcję tej struktury obejmującą trzy główne części składowe -- SortedStore, HashStore oraz LogStore, jak również metodę haszowania kukułczego, która jest w niej wykorzystywana. Zaimplementuję tę strukturę danych i zbadam empirycznie jej wydajność. Porównam ją z wynikami teoretycznymi i doświadczalnymi, przedstawionymi przez jej twórców.}
\englishabstract{\indent SILT (Small Index Large Table) is memory-effective high-performance associative array. It is an alternative for other popular data structures that solve key-value storage problem. It was developed for purpose of reducing and balancing memory usage and computation time. In my thesis I would like to focus on theoretical analysis of this data structure and comparison with \cpp\ standard library's function \texttt{std::unordered\_map}. As part of analysis I will describe its construction consisting of three main parts -- SortedStore, HashStore, LogStore and cuckoo hashing method. I will implement this structure and empirically examine its performance. I~will compare it with theoretical and experimental results, shown by its authors.}
\author{Paweł Guzewicz}
\advisor{dr hab. Marek Piotrów}
\transcriptnum{263664}
\advisorgen{dr. hab. Marka Piotrowa}
%\usepackage{graphicx,listings,amsmath,amssymb,amsthm,amsfonts,tikz}
%\theoremstyle{definition} \newtheorem{definition}{Definition}[chapter]
%\theoremstyle{remark} \newtheorem{remark}[definition]{Observation}
%\theoremstyle{plain} \newtheorem{theorem}[definition]{Theorem}
%\theoremstyle{plain} \newtheorem{lemma}[definition]{Lemma}
%\renewcommand \qedsymbol {\ensuremath{\square}}

\begin{document}
	\chapter{Wprowadzenie}
		\section{Przeznaczenie i zastosowanie}
			Small Index Large Table (SILT) to złożona struktura danych zaprojektowana z~myślą o efektywnej realizacji operacji słownikowych
			\begin{enumerate}
				\item
					Wstawiania: \texttt{insert(Key key, Value value)}
				\item
					Usuwania: \texttt{remove(Key key)}
				\item
					Podglądu: \texttt{lookup(Key key)}
			\end{enumerate}
			gdzie \texttt{Key} oraz \texttt{Value} oznaczają odpowiednio typ klucza i typ wartości.\\
			\indent W założeniu ma ona szybko obsługiwać powyższe operacje przy zrównoważonym zużyciu pamięci. Struktura ta ma być bowiem wykorzystywana w przypadku konieczności przechowywania dużej ilości danych. Podobnie jak w strukturach bazo-danynowych, także tu, kluczowym czynnikiem jest czas dostępu do pamięci dyskowej. (Przyjmujemy dla celów analizy, że jest to pamięć flash, podobnie jak w oryginalnej pracy \cite{SILT}.) Tablica SILT jest zaprojektowana tak, aby oszczędnie korzystać z tejże pamięci. Znajduje swe zastosowanie przy przetwarzaniu dużych danych (Big data) jako wariant bazy typu NoSQL.
		\section{Postulaty, koncepcje i cechy}
			Oczekujemy realizacji wymienionych operacji słownikowych w asymptotycznym czasie stałym. Oczywiście należy brać pod uwagę możliwość wykorzystania cache'u procesora, który zmniejsza czas dostępu do danych i jednocześnie długie czasy dostępu do pamięci zewnętrznej (pamięć flash lub dysk twardy), których chcemy uniknąć.\\
			\indent Uwzględniając sprzętowe aspekty należy stworzyć strukturę danych dopasowaną możliwie najlepiej do podanych operacji. Nie jest to łatwe, gdy dysponujemy jednym rodzajem magazynu danych. Rozwiązaniem jest połączenie kilku różnych struktur w~jedną. W każdej z tych podstruktur nacisk kładziony jest na inny aspekt. W~wyniku otrzymujemy kompromisowe rozwiązanie, które okazuje się być bardzo wydajne.\\
			\indent Innym celem, który chcemy osiągnąć jest skalowalność przy ograniczonym zużyciu pamięci przypadającej na jeden klucz.
		\section{Ogólny schemat działania}
			SILT składa się z trzech podstruktur: LogStore, kilku egzemplarzy HashStore i~SortedStore, które są ze sobą powiązane.\\
			\indent Pierwsza z nich odpowiada za szybką obsługę operacji wstawiania i usuwania. Wykorzystuje ona znaną technikę dopisywania kolejnych elementów (par klucz-wartość) na końcu dziennika (logu) -- pliku w pamięci flash. Jest to wysoce wydajne rozwiązanie, lecz nie zapewnia możliwości wyszukiwania, gdyż klucze są umieszczone w pliku w kolejności wstawiania. Do tego celu LogStore w~pamięci RAM przechowuje tablicę z haszowaniem, opartą o metodę haszowania kukułczego.\\
			\indent Podstruktury drugiego rodzaju (HashStore) są skompresowaną wersją LogStore. Nowa jednostka powstaje w wyniku konwersji przepełnionego LogStore'u. HashStore'y również zawierają tablice z haszowaniem pozbawione indeksu, przez co efektywniej wykorzystują pamięć. Ich jedynym zadaniem jest obsługa operacji podglądu, wobec czego mogą być strukturami tylko do odczytu. Wszystkie dane (tablica z haszowaniem) trzymane są w pamięci flash, a w pamięci RAM zostaje umieszczony filtr, który wstępnie określa, czy dany klucz może znajdować się w tym magazynie (przy niskim odsetku fałszywych pozytywnych wskazań).
			%\indent SortedStore
	\chapter{Struktura danych Small Index Large Table}
		\section{Przepływ danych wewnątrz struktury}
		\section{LogStore}
		\section{HashStore}
		\section{SortedStore}
		\section{Rozszerzenia}
	\chapter{Analiza teoretyczna}
		\section{Koszt operacji słownikowych}
		\section{Haszowanie kukułcze}
	\chapter{Analiza porównawcza}
		\section{Opis eksperymentu}
		\section{Wyniki}
	\chapter{Wnioski}
		\section{Teoretyczne własności}
		\section{Praktyczne osiągnięcia}
	\begin{thebibliography}{3}
		\bibitem{SILT} Hyeontaek Lim, Bin Fan, David G. Andersen, Michael Kaminsky - SILT: A Memory-Efficient, High-Performance Key-Value Store
		\bibitem{CH} Rasmus Pagh, Flemming Friche Rodler - Cuckoo hashing
		\bibitem{CHwS} Adam Kirsch, Michael Mitzenmacher, Udi Wieder - More Robust Hashing: Cuckoo Hashing with a Stash
	\end{thebibliography}
\end{document}
